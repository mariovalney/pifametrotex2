%%%%%%%%%%%%%%%%%%%%%%%%%%%%%%%%%%%%%%%%%%%%%%%%%%%%%%%%%%%%%%%%%%%%%%%%
%% Customizações do abnTeX2 (https://www.abntex.net.br/)              %%
%% para os Projetos Integradores da Fametro                           %%
%%                                                                    %%
%% This work may be distributed and/or modified under the             %%
%% conditions of the LaTeX Project Public License, either version 1.3 %%
%% of this license or (at your option) any later version.             %%
%% The latest version of this license is in                           %%
%%   http://www.latex-project.org/lppl.txt                            %%
%% and version 1.3 or later is part of all distributions of LaTeX     %%
%% version 2005/12/01 or later.                                       %%
%%                                                                    %%
%% This work has the LPPL maintenance status `maintained'.            %%
%%                                                                    %%
%% The Current Maintainer of this work is Mário Valney                %%
%%                                                                    %%
%% Project available on: https://github.com/mariovalnet/pifametrotex2 %%
%%                                                                    %%
%% Further information about abnTeX2                                  %%
%% are available on https://www.abntex.net.br/                        %%
%%                                                                    %%
%%%%%%%%%%%%%%%%%%%%%%%%%%%%%%%%%%%%%%%%%%%%%%%%%%%%%%%%%%%%%%%%%%%%%%%%

\documentclass[
    % -- opções da classe memoir --
    a4paper,                % Tamanho do papel
    12pt,                   % Tamanho da fonte
    oneside,                % Usada para impressão em apenas uma face do papel / twoside
    % -- opções da classe abntex2 --
    chapter=TITLE,          % Todos os capítulos devem ter caixa alta
    section=Title,          % Todas as seções devem ter caixa alta
    %subsection=TITLE,      % títulos de subseções convertidos em letras maiúsculas
    %subsubsection=TITLE,   % títulos de sub-subseções convertidos em letras maiúsculas
    % -- opções do pacote babel --
    english,                % Hifenizações em inglês
    spanish,                % Hifenizações em espanhol
    brazil                  % Idioma padrão do documento
]{abntex2}

%%%%%%%%%%%%%%%%%%%%%%%%%%%%%%%%%%%%%%%%%
%%              Pacotes                %%
%%%%%%%%%%%%%%%%%%%%%%%%%%%%%%%%%%%%%%%%%

% Importações de pacotes - Básicos
% \usepackage{lmodern}          % Usa a fonte Latin Modern
\usepackage{mathptmx}           % Usa a fonte Times New Roman
\usepackage[T1]{fontenc}        % Selecao de codigos de fonte.
\usepackage[utf8]{inputenc}     % Codificacao do documento (conversão automática dos acentos)
\usepackage{indentfirst}        % Indenta o primeiro parágrafo de cada seção.
\usepackage{color}              % Controle das cores
\usepackage{graphicx}           % Inclusão de gráficos
\usepackage{microtype}          % para melhorias de justificação

% Importações de pacotes - Canônicos abnTeX2
\usepackage{lipsum}             % Geração de dummy text

% Importações de pacotes - Citações
\usepackage[brazilian,hyperpageref]{backref}                                                                                % Paginas com as citações
\usepackage[alf,abnt-emphasize=bf,bibjustif,recuo=0cm,abnt-etal-cite=3,abnt-etal-list=0,abnt-etal-text=it]{abntex2cite}     % Citações padrão ABNT

% Importações de pacotes - Misc
\usepackage{amsfonts, amssymb, amsmath}             % Fonte e símbolos matemáticos
\usepackage{booktabs}                               % Comandos para tabelas
\usepackage{verbatim}                               % Texto é interpretado como escrito no documento
\usepackage{multirow, array}                        % Múltiplas linhas e colunas em tabelas
\usepackage{listings}                               % Utilizar código fonte no documento
\usepackage{microtype}                              % Para melhorias de justificação?
\usepackage{tocloft}                                % Permite alterar a formatação do Sumário
\usepackage{etoolbox}                               % Usado para alterar a fonte da Section no Sumário
% \usepackage[nogroupskip,nonumberlist,acronym]{glossaries}   % Permite fazer o glossario
\usepackage{caption}                                % Altera o comportamento da tag caption
\usepackage{ifthen}                                 % Inclui condicionais

% Importações de pacotes - Fametro
\usepackage{lib/pifametrotex2}          % Formato da Fametro

%%%%%%%%%%%%%%%%%%%%%%%%%%%%%%%%%%%%%%%%%
%%           Configurações             %%
%%%%%%%%%%%%%%%%%%%%%%%%%%%%%%%%%%%%%%%%%

% Margen da legenda da figura e da tabela
\captionsetup{justification=raggedright,singlelinecheck=false,skip=0pt}